%!TEX root = ../../main.tex

\begin{figure}[!htb]
\begin{center}
\begin{tikzpicture}[<->,>=stealth,auto,node distance=3cm,thick,
CHR/.style={circle,draw,thick,minimum size=0.8cm,font=\sffamily\bfseries},
IDV/.style={rectangle,draw=gray,thick,inner sep=6mm}]
\node[CHR,fill=Ivory2] (1) {1};
\node[CHR,fill=Azure2] (2) [right of=1] {2};
\node[CHR,fill=Ivory2] (3) [right of=2] {1};
\node[CHR,fill=Azure2] (4) [right of=3] {2};
\node[IDV,label={[font=\sffamily]below:Individual A},fit=(1) (2)] {};
\node[IDV,label={[font=\sffamily]below:Individual B},fit=(3) (4)] {};
\draw [<->] (1) to [out=18,in=162] (2);
\draw [<->] (1) to [out=36,in=144,looseness=0.666] (3);
\draw [<->] (1) to [out=54,in=126,looseness=0.666] (4);
\draw [<->] (2) to [out=18,in=198] (3);
\draw [<->] (3) to [out=342,in=198] (4);
\draw [<->] (2) to [out=324,in=216,looseness=0.666] (4);
\end{tikzpicture}
\end{center}
\Caption{Illustration of the possible shared haplotypes in two individuals}
{what the hell is going on here. i have no idea!}
{fig:six_chrom_pairs}
% \Caption{Illustration of the possible shared haplotypes in two individuals}
% {Given \n{2} diploid individuals, each of the \n{4} chromosomes may share a haplotype identical by descent.}
% {Given \n{2} diploid individuals, each of the \n{4} chromosomes may share a haplotype identical by descent.
% As such, it is possible that the inferred IBD regions may overlap.}
% {fig:six_chrom_pairs}
% \vspace{-5pt}
% \hrulefill%
\end{figure}
